% !TeX root = ../tese_lncc.tex
\chapter[Introdução]{Introdução}\label{cap_intro}

\section{Desenvolvimento Racional de Fármacos Baseado em Estrutura}
O Desenvolvimento racional de fármacos é uma metodologia que auxilia na diminuição do custo, tempo e do número de falhas que o desenvolvimento de um novo fármaco pode tomar.

\section{Docking}
\section{Docking baseado em campos de força}
\section{Virtual Screening}
\section{Docking baseado em grade}

\chapter{DockThor}
\section{Histórico}
\section{Implementação}

\chapter{Computação paralela}
\section{Escalabilidade}
\section{Arquitetura CPU}
\section{OpenMP}
\section{MPI}
\section{GPU}
\subsection{Arquitetura GPU}


\chapter{Revisão bibliográfica Algoritmos evolutivos e Docking HPC}

\chapter{Objetivos}
\section{DockThor residente em GPU}

\chapter{Metodologia}
\section{Princípio da localidade e acesso coalescente à memória: Struct of Array (SoA)}
\section{Grade DockThor em SoA}
\section{Steady State em SoA}
\section{Geracional}
\section{Geracional em GPU}

\chapter{Resultados}
\section{Speedup Grade}
\subsection{Speedup Grade com Sequencial AoS}
\subsection{Speedup Grade com Sequencial SoA}
\subsection{Speedup Grade com OpenMP}
\subsection{Speedup Grade com OpenCL}

\section{Acurácia SteadyState SoA}
\section{Speedup SteadyState SoA/AoS}
\section{Acurácia Geracional}
\section{Speedup Geracioal CPU}
\section{Speedup Geracioal GPU}


\chapter{Conclusão e trabalhos futuro}